\chapter{Árvore Heap}
\label{ch:heap} % This how you label a chapter and the key (e.g., ch:into) will be used to refer this chapter ``Introduction'' later in the report. 

% the key ``ch:into'' can be used with command \ref{ch:intor} to refere this Chapter.
% 
\section*{Introdução}
A Heap (ou Binary Heap) é uma estrutura de dados que representa uma árvore binária completa ou quase-completa. Nesse tipo de árvore, todos os níveis, exceto possivelmente o último, estão preenchidos, da esquerda para a direita.

Existem duas variações de Binary Heap: Max Heap e Min Heap. No Max Heap, os pais têm valores maiores que ou iguais aos dos filhos, enquanto no Min Heap, os pais têm valores menores que ou iguais aos dos filhos.

Nesse projeto, a Heap foi implementada em Rust e foram utilizados testes para verificar as funções de Alteração de Prioridade, Inserção, Remoção (da raiz) e Construção das Heaps. Tais testes podem ser vistos funcionando no vídeo enviado pelo grupo.

\section*{HeapSort}

HeapSort é um algoritmo de ordenação que usa uma Árvore Heap para classificar os elementos. Ele opera construindo o heap, extraindo repetidamente a raiz e reconstituindo a heap.

Para isso, algumas funções auxiliares são necessarias como: construir(tranforma uma lista em Heap levando em conta suas propriedades e características), subir(move um elemento para níveis superiores) e descer (move um elemento para níveis inferiores).

Dessa forma, estabelecemos o pseudocódigo da HeapSort como:

\begin{algorithm}
	\caption{HeapSort}
	\label{algo:heap_sort}
	\begin{algorithmic}[1]
		\Require{Lista $A = A_1, A_2, \ldots, A_n$}
		\Ensure{Lista $A$ ordenada}
		\Statex
		\Function{HeapSort}{$A$}
        \State \textbf{construir(A)}
        \State {$tamanho \gets A.tamanho$
		\For{$j \gets tamanho$ to $1$}
        \State \textbf{troque} $A[1]$ por $A[i]$
        \State tamanho = tamanho - 1;
		\State \textbf{descer} (H, 1);
		\EndFor
		\EndFunction
	\end{algorithmic}
\end{algorithm}
\FloatBarrier

\section*{Comparação com outros Sorts}

Agora, vamos comparar o resultado dos testes feitos no HeapSort com o comportamento de outros algoritmos de ordenação vistos em nosso último trabalho. Os resultados abaixo foram obtidos utilizando o ambiente computacional mencionado no Capítulo 1.

\begin{table}[h!]
	\centering
	\caption{Tabela de resultados dos sorts}
	\label{tab:sorts_result}
	\begin{tabular}{lrrrr}
		\toprule
		Tipo de ordenação   & n = 10000   & n = 1000000  & n = 100000000 \\
		\midrule
	ITE Bubble Sort & 30.09409ms & 8.141286072s & 918.838945524s \\
    REC Bubble Sort & 41.421701ms & 9.076120126s & 957.590337298s \\
    REC Quick Sort  & 376.014µs & 4.472229ms   & 48.459402ms \\
    ITE Quick Sort  & 388.286µs & 4.762031ms   & 50.682197ms \\
    REC Merge Sort  & 760.342µs & 8.49751ms    & 93.844578ms \\
    ITE Merge Sort  & 467.715µs & 5.939955ms   & 67.584511ms \\
		\midrule
		Heap Sort       & 595.914µs & 7.452145ms   & 98.36624ms \\
		\bottomrule
	\end{tabular}
\end{table}

OBS: "ITE" significa Iterativo e "REC" significa Recursivo  

Vemos que, dentre os algoritmos analisados, o Heap Sort apresenta uma eficiência consistente, superando significativamente os tempos de execução dos algoritmos Bubble Sort (tanto iterativo quanto recursivo), que se mostram inadequados para listas maiores devido ao alto custo computacional. Apesar de ligeiramente superado pelos Quick Sorts (iterativo e recursivo) em todos os casos apresentados, o Heap Sort mantém resultados próximos em listas pequenas e médias. Quanto ao Merge Sort, observa-se uma proximidade significativa em eficiência com o Heap Sort: a versão recursiva apresenta desempenho semelhante, enquanto a versão iterativa consegue uma leve vantagem, especialmente em listas menores. Isso evidencia que o Heap Sort, apesar de não ser o mais rápido em todas as situações, continua sendo uma escolha confiável para ordenações eficientes.

% \input{chapters/03_sections/01_heapify}
% \newpage
%
