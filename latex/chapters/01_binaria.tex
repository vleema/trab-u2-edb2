\documentclass[a4paper,12pt]{article}
\usepackage[utf8]{inputenc}
\usepackage{amsmath}

\usepackage{geometry}
\usepackage{tikz}
\geometry{a4paper, margin=1in}

\begin{document}

\section*{Árvore Binária}

Para criar uma árvore binária pouco desequilibrada a partir de uma lista de dados de entrada, podemos usar a estratégia de \textbf{divisão e conquista}. A ideia é organizar os dados de entrada, selecionar o elemento do meio como raiz e repetir esse processo para as sub-árvores da esqueda e direita.

\subsection*{Estratégia: Construção Balanceada com Lista Ordenada}

\begin{enumerate}
    \item \textbf{Ordenar os dados de entrada:}
    \begin{itemize}
        \item Se os dados não estiverem ordenados, ordene-os. Isso garante que a árvore seja uma \textit{árvore binária de busca}.
    \end{itemize}

    \item \textbf{Escolher o elemento do meio como raiz:}
    \begin{itemize}
        \item Divida a lista ao meio. O elemento central vai ser a raiz da árvore ou subárvore.
    \end{itemize}

    \item \textbf{Repetir de forma recursiva:}
    \begin{itemize}
        \item Para os elementos à esquerda do meio, crie a subárvore esquerda.
        \item Para os elementos à direita do meio, crie a subárvore direita.
    \end{itemize}

    \item \textbf{Continuar até que a lista esteja vazia:}
    \begin{itemize}
        \item A recursão terminará quando não houverem mais elementos na lista.
    \end{itemize}
\end{enumerate}

\subsection*{Exemplo: Construção da Árvore}

Considere os seguintes dados iniciais:

\[
80, \; 20, \; 50, \; 30, \; 90, \; 70, \; 10, \; 40, \; 60
\]

\noindent
Após ordenar, temos:

\[
10, \; 20, \; 30, \; 40, \; 50, \; 60, \; 70, \; 80, \; 90
\]

O elemento central (\textbf{50}) será a raiz.  
Agora, dividiremos a lista em duas partes:
\begin{itemize}
    \item Subárvore esquerda: \(10, 20, 30, 40\)
    \item Subárvore direita: \(60, 70, 80, 90\)
\end{itemize}

Construindo visualmente:

\begin{center}
\begin{tikzpicture}[
  level distance=1.5cm,
  level 1/.style={sibling distance=6cm},
  level 2/.style={sibling distance=3cm},
  level 3/.style={sibling distance=1.5cm},
  edge from parent/.style={draw, -latex},
  every node/.style={circle, draw, minimum size=8mm, font=\footnotesize}
]

% Raiz
\node {50}
  child {node {30}
    child {node {20}
      child {node {10}}
      child[missing]
    }
    child {node {40}}
  }
  child {node {70}
    child {node {60}}
    child {node {80}
        child[missing]
        child {node {90}}
    }
  };

\end{tikzpicture}
\end{center}

\subsection*{Explicação do Diagrama}
\begin{enumerate}
    \item O nó \textbf{50} é escolhido como raiz da árvore, pois é o elemento central da lista.
    \item A subárvore esquerda contém os números menores que 50 (\textbf{10, 20, 30, 40}):
    \begin{itemize}
        \item O elemento central (\textbf{30}) é escolhido como a raiz da subárvore esquerda.
        \item Os elementos menores que 30 (\textbf{10, 20}) formam sua subárvore esquerda.
        \item O elemento \textbf{40} forma a subárvore direita de 30.
    \end{itemize}
    \item A subárvore direita contém os números maiores que 50 (\textbf{60, 70, 80, 90}):
    \begin{itemize}
        \item O elemento central (\textbf{70}) é escolhido como a raiz da subárvore direita.
        \item O número \textbf{60} forma sua subárvore à esquerda.
        \item Os números \textbf{80} e \textbf{90} formam a subárvore à direita.
    \end{itemize}
\end{enumerate}

\end{document}
